%%%%%%%%%%%%%%%%%%%%%%%%%%%%%%%%%%%%%%%%%%%%%%%%%%%%%%%%%%%%%%%%%%%%%%%%%%
%
% Ph.D. dissertation manuscript
% Global LaTeX macros
%
% Brad Student (January 2005)
% College of William and Mary
% Department of Physics
% Prof. Sharp Mind, advisor
%
% $Id: Diss-title.tex,v 1.5 2005/01/14 22:32:00 wirawan Exp $
%
% Based on Paul King and Andrew Norman's template (modified)
%
%%%%%%%%%%%%%%%%%%%%%%%%%%%%%%%%%%%%%%%%%%%%%%%%%%%%%%%%%%%%%%%%%%%%%%%%%%

% Simple equation macros (note how the align or align* environment is used
% in AMS-math documentation, if you're going to use this).
\newcommand\eq[1]                              % single unlabeled equation
{
\begin{align*}
#1
\end{align*}
}

\newcommand\eql[2] % single labeled equation
{
\begin{equation}\label{#1}
\begin{split}
#2
\end{split}
\end{equation}
}

\newcommand\eqnl[2]        % single labeled equation (maunya sih, centered
{%                     equation array dg satu label saja (apakah bisa???))
\begin{equation}\label{#1}
%\begin{split}
#2
%\end{split}
\end{equation}
}

% WARNING: eqsl is REDEFINED!
\newcommand\eqsl[1]                            % multiply labeled equation
{
\begin{align}
#1
\end{align}
}

\newcommand\eqssl[2]                      % multiply labeled SUB-equations
{
\begin{subequations}\label{#1}
\begin{align}
#2
\end{align}
\end{subequations}
}

% FORMATTING SHORTCUTS:
\newcommand\prg[1]     {\texttt{#1}}                   % computerish stuff

% Formatting for captions:
% Usage:
% \xcaption{Label}{First caption included in TOC.}{More caption text.}
% or
% \xcaption[Caption for TOC.]{Label}{First caption text.}{More caption text.}
\newcommand\xcaption[4][] {
    \ifthenelse{\equal{#1}{}}
               {\caption[#3]{\label{#2}#3 #4}}
               {\caption[#1]{\label{#2}#3 #4}}
}
% The following is needed because caption is too close to the table:
\newcommand\TableCaptionSkip {\vskip 0.4cm}
% Table caption: similar to \xcaption above
\newcommand\tcaption[4][] {
    \xcaption[#1]{#2}{#3}{#4}
    \TableCaptionSkip
}
% Autocentering for graphics:
\newcommand\IncludeGraphics[2][] {
    \begin{center}
    \includegraphics[#1]{#2}
    \end{center}
}
\newcommand\FIGURE[1]  {figures/#1}     % -- if specific-plotdir is wanted

% Referencing shortcuts:
\providecommand\eqref[1] {(\ref{#1})}  %if your stylesheet doesn't have it
\newcommand\Appendix[1]{Appendix~\ref{#1}}
\newcommand\Chapter[1] {Chapter~\ref{#1}}
\newcommand\ChapSec[2] {Chapter~\ref{#1}, \Sec{#2}}
\newcommand\Eq[1]      {Eq.~\eqref{#1}}
\newcommand\Eqs[1]     {Eqs.~\eqref{#1}}
\newcommand\Fig[1]     {Fig.~\ref{#1}}
\newcommand\Sec[1]     {Sec.~\ref{#1}}
\newcommand\Ref[1]     {Ref.~\onlinecite{#1}}
\newcommand\Refs[1]    {Refs.~\onlinecite{#1}}
\newcommand\Cite[1]    {~\cite{#1}}                % user-modifiable \cite

% Coloring and marking
\definecolor{Red}{rgb}{0.8,0.2,0.2}%red
\definecolor{Green}{rgb}{0.08,0.60,0.08}%green
\definecolor{Blue}{rgb}{0.3,0.3,1.0}%blue
\definecolor{dkBlue}{rgb}{0.1,0.1,0.5}%dark blue
\definecolor{Black}{rgb}{0,0,0}
\definecolor{Gold}{rgb}{0.60,0.60,0.08}%gold-like color
\definecolor{Remarks}{rgb}{1,0.3,0.3}%red
\definecolor{Extra}{rgb}{0.2,0.2,1}%blue

% DRAFTING utility
\newcommand\COMMENTED[1] {}

% KE - added following Macros
\newcommand\ket[1]     {|{{#1}}\rangle}
\newcommand\bra[1]     {\langle{{#1}}|}

\newcommand\REMARKS[1]   {\textbf{\textcolor{Red}{[#1]}}}
%\newcommand\REMARKS[1]   {}
\newcommand\REMARKSBLUE[1]   {\textbf{\textcolor{Blue}{[#1]}}}
% You are certainly welcome to customize the macros above (or erase the
% ones that you don't like). Add your own macros here, too.
